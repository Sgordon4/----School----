\documentclass[12pt]{article}
\usepackage{amsmath}
\usepackage{amssymb}
\usepackage{graphicx}
\usepackage{hyperref}
\usepackage{multicol}
\usepackage[latin1]{inputenc}
\usepackage{listings}
\usepackage{scrextend}

\title{EE 330\\Section 5\\Homework 4}
\author{Sean Gordon}
%\date{09/09/2019}

\begin{document}
\maketitle
This is just algebra busy work.\\\indent There is little learning being done here.\\


\noindent 1) Area of wafer = $\pi (300/2)^2 = 70685.8mm^2$\\
\# Chips/Wafer: $70685.8/50 = 1413\ chips$\\
Hours/Year: $365\ days/yr * 24\ hrs/day = 8760 hrs$ \\
\begin{multicols}{2}

\noindent 248 nm Machine:\\
Wafers/Year: $80\ wafers/hr * 8760\ hr/yr = 700,800\ wafers/yr$\\ \\
Cost/Wafer: $\$10M/700,800=\$14.27$\\ \\
Cost/Chip: $\$14.27 / 1413 = \$0.01$


\columnbreak
193 nm Machine:\\
Wafers/Year: $20\ wafers/hr * 8760\ hr/yr = 175,200\ wafers/yr$\\ \\
Cost/Wafer: $\$40M/175,200=\$228.31$\\ \\
Cost/Chip: $\$228.31 / 1413 = \$0.162$

\end{multicols} 
\noindent Difference = $\$0.162 - \$0.01 = \$0.152$\\\\


\noindent 2) Dielectrics: $SiO_2 = 3.9, HfO_2 = 25$.
Thickness must be proportional to dielectric, therefore $t_{HfO_2} = 25/3.9 * 2 = 12.82 nm$\\

\noindent 3) Vol $SiO_2 = .044nm^3\indent \indent 25 A \to 2.5 nm$\\
$7nm * 14nm * 2.5nm = 245 nm ^3 = 5568\ molecules$\\

\noindent 4) Resistivity of Aluminum = $2.8*10^{-8}\Omega m$\\


\noindent 5) Silver. Expensive with high electron migration potential.\\ 


\noindent 6) 300mm wafer thickness = 775 +/- 25um, + 150um for saw = 925um\\
$2m / (925 +/- 25um ) = 2105 \to 2222\ wafers$\\

\noindent 7)





\end{document}
