\documentclass[12pt]{article}
\usepackage{amsmath}
\usepackage{amssymb}
\usepackage{graphicx}
\usepackage{hyperref}
\usepackage{multicol}
\usepackage[latin1]{inputenc}
\usepackage{listings}
\usepackage{scrextend}


\title{EE 330\\Section 5\\Homework 4}
\author{Sean Gordon}
%\date{09/09/2019}

\begin{document}
\maketitle
This is just algebra busy work.\\\indent There is little learning being done here.\\


\noindent 1) Area of wafer = $\pi (300/2)^2 = 70685.8mm^2$\\
\# Chips/Wafer: $70685.8/50 = 1413\ chips$\\
Hours/Year: $365\ days/yr * 24\ hrs/day = 8760 hrs$ 
\begin{multicols}{2}

\noindent 248 nm Machine:\\
Wafers/Year: $80\ wafers/hr * 8760\ hr/yr = 700,800\ wafers/yr$\\ \\
Cost/Wafer: $\$10M/700,800=\$14.27$\\ \\
Cost/Chip: $\$14.27 / 1413 = \$0.01$


\columnbreak
193 nm Machine:\\
Wafers/Year: $20\ wafers/hr * 8760\ hr/yr = 175,200\ wafers/yr$\\ \\
Cost/Wafer: $\$40M/175,200=\$228.31$\\ \\
Cost/Chip: $\$228.31 / 1413 = \$0.162$

\end{multicols} 
\noindent Difference = $\$0.162 - \$0.01 = \$0.152$\\\\


\noindent 2) Dielectrics: $SiO_2 = 3.9, HfO_2 = 25$.
Thickness must be proportional to dielectric, therefore $t_{HfO_2} = 25/3.9 * 2 = 12.82 nm$\\

\noindent 3) Vol $SiO_2 = .044nm^3\indent \indent 25 A \to 2.5 nm$\\
$7nm * 14nm * 2.5nm = 245 nm ^3 = 5568\ molecules$\\

\noindent 4) Resistivity of Aluminum = $2.8*10^{-8}\Omega m$,\indent $R = \frac{\rho*l}{w*t}$\\
$R_{Al} = \frac{(2.8*10^{-8})*(200*10^{-6})}{(60*10^{-9})*(60*10^{-9})} = 1555.5\Omega$\\


\noindent 5) Silver. Expensive with high electron migration potential.\\ 


\noindent 6) 300mm wafer thickness $= 775 \pm 25um$, $+ 150um$ for saw $= 925um$\\
$2m / (925 \pm 25um ) = 2105 \to 2222$ wafers.\\


\noindent 7) What is this.\\


\noindent 8a) Length = 1um Width = 3um\\
b) Photoresist under-exposed $\to length = .8 um$\\
A combination of both issues leaves width unchanged = 3um.\\
c) Width still unchanged, but now issues cancel out so length is unchanged so that length = 1um.\\


\noindent 9) $\rho_{Al} = 2.8*10^{-8}\Omega m,\ \ \  R = \frac{\rho*l}{w*t}$\\
t = $ \frac{ (2.8*10^{-8}) * (250*10^{-6}) }{ (2*10^{-6}) * 25 } = .14um $\\
Sheet Resistance = $\frac{\rho}{t} = \frac{2.8*10^{-8}}{.14*10^{-6}} = .2\Omega$ \\


\noindent 10)   $\rho_{Cu} 1.68*10^{-8}\Omega m,\ \ \  R = \frac{\rho*l}{w*t}$\\
$ l = \frac{ (2*10^{-6}) * (.14*10^{-6}) * 25 }{(1.68*10^{-8})} = 417um $\\

\noindent 11) Thermal silicon growth uses silicon from substrate, approx .47x full height of oxide. Therefore, if oxide is 5000A, $W_{height}$ increase is $5000*(1-.47) = 2650A$  \\

\noindent 12) Sheet resistance given, $R_{poly} = 23.2\ \Omega/sq\ somethings$ (thanks for leaving that blank btw, very helpful), and $R_{p+} = 106.7\ \Omega/sq\ somethings$.\\
As the units are conveniently obfuscated, I will be assuming they are $sq\ um$.\\\\
5k resistor: Poly $\to 5000/23.2= 215.52um $ len,\\ 
\indent \indent \indent \ \ \ P+ $\to 5000/106.7 = 46.86um$ len.\\
Min size Poly = .2um\ x\ .2um, Min size P+ = .3um\ x\ .3um\\
Min Poly serpentine = $.2*.2*215.52 = 8.62um^3$.\\ 
Min P+ serpentine\ \ = $.3*.3*46.86\ \ = 4.22 um^3$.\\
$8.62/4.22 \to 2.04x$ area increase if using Poly.

\iffalse

\noindent 12) Noted in lecture 7, $R_{poly} = 25\Omega \pm25\%$, and $R_{p+} = 80\Omega \pm 25$.\\
However, the units are conveniently obfuscated, so I will be assuming they are um.\\
5k resistor: Poly $\to 5000/25= 200um $ len, P+ $\to 5000/80 = 62.5um$ len.\\
Min size Poly = .2um\ x\ .2um, Min size P+ = .3um\ x\ .3um\\
Min Poly serpentine = $.2*.2*200 = 8um^3$.\\ Min P+ serpentine = $.3*.3*62.5 = 5.625 um^3$.\\
$8/5.625 \to 1.42x$ size if using Poly.\\\\

\fi







\end{document}










































