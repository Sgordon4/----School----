\documentclass[12pt]{article}
\usepackage{amsmath}
\usepackage{amssymb}
\usepackage{graphicx}
\usepackage{hyperref}
\usepackage{multicol}
\usepackage[latin1]{inputenc}
\usepackage{listings}
\usepackage{scrextend}
\usepackage{tabularx}

\usepackage{subcaption}

\title{CprE 308\\Section 3\\Lab7\&8}
\author{Sean Gordon\\Sgordon4}
%\date{November 12, 2019}

\begin{document}
\maketitle

\noindent These labs focused on decoding sections of the FAT filesystem, giving a deeper understanding of the setup of different headers. The example file image provides a realistic example, with the printed results helping to cement ideas.\\

\noindent I had never done any work with filesystems before, and was suprised with how smoothly these labs went. For some reason I got super stuck on printing out the root directory sizes though, which took a fair amount of time to fix. Other than that, I feel like this was a successful foray into filesystems and their headers.\\\\\\\\

Exercise 1: 
\begin{center}
 \begin{tabularx}{1\textwidth} { 
  | >{\raggedright\arraybackslash}X 
  | >{\centering\arraybackslash}X 
  | >{\centering\arraybackslash}X | }
 \hline
  & Hex & Decimal \\\hline
 Bytes per sector & 0x0200 & 512 \\\hline
 Sectors per cluster & 0x10 & 16 \\\hline
 Max Number of Root directory entries & 0x00e0 & 224\\ \hline
 Sectors per FAT & 0x0001 & 1\\  [1ex] 
 \hline
\end{tabularx}
\end{center}


\end{document}