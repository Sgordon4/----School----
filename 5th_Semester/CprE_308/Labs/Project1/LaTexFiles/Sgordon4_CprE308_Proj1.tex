\documentclass[12pt]{article}
\usepackage{amsmath}
\usepackage{amssymb}
\usepackage{graphicx}
\usepackage{hyperref}
\usepackage{multicol}
\usepackage[latin1]{inputenc}
\usepackage{listings}
\usepackage{scrextend}
\usepackage{changepage} %Adjustwidth


% Used for code blocks ----------------------------------------------------------------
\usepackage{color}
\usepackage{xcolor}
\usepackage{listings}

\usepackage{caption}
\DeclareCaptionFont{white}{\color{white}}
\DeclareCaptionFormat{listing}{\colorbox{gray}{\parbox{\textwidth}{#1#2#3}}}
\captionsetup[lstlisting]{format=listing,labelfont=white,textfont=white}
% -----------------------------------------------------------------------------------------


\title{CprE 308\\Project 1}
\author{Sean Gordon}
%\date{09/29/2019}

\begin{document}
\maketitle

This lab focused on the creation of a simple shell, employing forking practices we learned earlier, along with a large amount of string manipulation.\\
I personally haven't used forking very often in my own projects, so the implementation of background processes without leaving a mess behind was a new challenge.\\
This was done by launching an observer by forking main, then launching the worker by forking the observer. In this way the observer could wait for a response while main continued unhindered.\\

While I left the extra credit options mostly unimplemented, the ideas behind them are easy to grasp.\\
The jobs command would be implemented using a linked list of all active child processes, adding and removing at appropriate times.\\
The output dump option would use a pipe and typical file creation functions.\\


\end{document}