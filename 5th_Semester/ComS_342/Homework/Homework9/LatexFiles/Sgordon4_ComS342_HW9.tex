\documentclass[12pt]{article}
\usepackage{amsmath}
\usepackage{amssymb}
\usepackage{graphicx}
\usepackage{hyperref}
\usepackage{multicol}
\usepackage[latin1]{inputenc}
\usepackage{listings}
\usepackage{scrextend}
\usepackage{changepage} %Adjustwidth
\usepackage[margin=1in]{geometry}

\usepackage{algorithm}
%\usepackage{algorithmic}
\usepackage{algpseudocode}
\usepackage{alltt}


% Used for code blocks ----------------------------------------------------------------
\usepackage{color}
\usepackage{xcolor}
\usepackage{listings}

\usepackage{caption}
\DeclareCaptionFont{white}{\color{white}}
\DeclareCaptionFormat{listing}{\colorbox{gray}{\parbox{\textwidth}{#1#2#3}}}
\captionsetup[lstlisting]{format=listing,labelfont=white,textfont=white}
% -----------------------------------------------------------------------------------------



\title{ComS 342\\Recitation 2, 10:00 Tuesday\\Homework 9}
\author{Sean Gordon}
%\date{09/09/2019}

\begin{document}
\maketitle


\hrulefill \\


\begin{center}
\noindent 1) Z = [1, 4, 6, 3, 6].
\end{center}


\hrulefill \\


\begin{multicols}{2}

\noindent 2a)\\\\
fib(0, 0).\\
fib(1, 1).\\\\
fib(N, Res) :-\\
\indent N $>$ 1,\\
\indent N1 is N-1,\\
\indent N2 is N-2,\\\\
\indent fib(N1, R1),\\
\indent fib(N2, R2),\\\\
\indent Res is R1+R2.\\

\columnbreak

\noindent 2b)\\\\
rev(L, Res) :-\\
\indent revHelper(L, [ ], Res).\\\\
revHelper([ ], Accum, Accum).\\
revHelper([H$|$T], Accum, Res) :-\\
\indent number(H),\\
\indent revHelper(T, [H$|$Accum], Res);\\\\
\indent revHelper(H, [ ], Temp),\\
\indent revHelper(T, [Temp$|$Accum], Res).\\

\end{multicols} 


\hrulefill \\
\pagebreak


\noindent 3a)\\
sentence([ ]).\\
sentence(S) :- s(S, [ ]).\\

\noindent s(L1, L2) :- f(L1, L2).\\
s(L1, L4) :- t(L1, L2), n(L2, L3), t(L3, L4).\\

\noindent f(L1, L5) :- if(L1, L2), b(L2, L3), \\
\ \ \ \ \ \ \ \ \ \ \ \ \ \ \ \ \ \ \ \ \ then(L3, L4), f\_to\_s(L4, L5).\\
\noindent f(L1, L7) :- if(L1, L2), b(L2, L3), then(L3, L4), \\
\ \ \ \ \ \ \ \ \ \ \ \ \ \ \ \ \ \ \ \ \ s(L4, L5), else(L5, L6), f\_to\_s(L6, L7).\\

\noindent f\_to\_s([X$|$\_], L2) :- s(X, L2).\\

\noindent b(L1, L4) :- t(L1, L2), e(L2, L3), t(L3, L4).\\

\noindent if([if$|$Tail], Tail).\\
then([then$|$Tail], Tail).\\
else([else$|$Tail], Tail).\\

\noindent t([x$|$Tail], Tail).\\
t([y$|$Tail], Tail).\\
t([z$|$Tail], Tail).\\
t([1$|$Tail], Tail).\\
t([0$|$Tail], Tail).\\

\noindent e([$<$$|$Tail], Tail).\\
e([$>$$|$Tail], Tail).\\

\noindent n([+$|$Tail], Tail).\\
n([-$|$Tail], Tail).\\
n([=$|$Tail], Tail).\\\\


\noindent 3b) Uhhh... sentence(X)? Exceeds memory due to S $<$-$>$ F recursion. I'm confused.\\

\noindent 3c) Yes, sub-goal order matters as they are executed sequentially.\\


\hrulefill \\


\noindent 4) Hmmm................................

\end{document}
















