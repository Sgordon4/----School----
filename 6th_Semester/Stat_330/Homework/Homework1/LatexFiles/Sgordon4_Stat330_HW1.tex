\documentclass[12pt]{article}
\usepackage{amsmath}
\usepackage{amssymb}
\usepackage{graphicx}
\usepackage{subcaption}
\usepackage{hyperref}
\usepackage{multicol}
\usepackage[latin1]{inputenc}
\usepackage{listings}
\usepackage{scrextend}

\usepackage{changepage} %Adjustwidth
\usepackage[margin=1in]{geometry}

\usepackage{algorithm}
%\usepackage{algorithmic}
\usepackage{algpseudocode}
\usepackage{alltt}


% Used for code blocks ----------------------------------------------------------------
\usepackage{color}
\usepackage{xcolor}
\usepackage{listings}

\usepackage{caption}
\DeclareCaptionFont{white}{\color{white}}
\DeclareCaptionFormat{listing}{\colorbox{gray}{\parbox{\textwidth}{#1#2#3}}}
\captionsetup[lstlisting]{format=listing,labelfont=white,textfont=white}
%-----------------------------------------------------------------------------------------



\title{Stat 330\\Homework 1}
\author{Sean Gordon}
%\date{09/09/2019}

\begin{document}
\maketitle


\hrulefill \\


\noindent 1)\\
\indent (a) $\Omega$ = \{T T T, \ T T H, \ T H T, \ T H H,\\ \ \ \
\indent \indent \indent \ \ \ \ H T T, \ H T H, \ H H T, \ H H H\}\\

(b) A = \{T T H, \ T H T, \ H T T\}\\
\indent \indent \ B = \{T T H, \ T H T, \ H T T, \ T T T\}\\
\indent \indent \ C = \{T H H, \ H H H\}\\

(c) $\overline{A}$ = \{T T T, \ T H H, \ H T H, \ H H T, \ H H H\}\\
\indent \indent A $\cup$ B = \{T T H, \ T H T, \ H T T, \ T T T\}\\
\indent \indent A $\cap$ B = \{T T H, \ T H T, \ H T T\}\\
\indent \indent A $\cap$ C = \{ \}\\


\hrulefill \\


\noindent 2)\\
\indent \indent A = \{1, 3, 5\}\\
\indent \indent B = \{1, 5, 10\}\\
\indent \indent $\overline{A}$ = \{2, 4, 6, 7, 8, 9, 10\}\\
\indent \indent $\overline{B}$ = \{2, 3, 4, 6, 7, 8, 9\}\\

(b) A $\cap$ B = \{1, 5\}\\
\indent \indent $\overline{A \cap B}$ = \{2, 3, 4, 6, 7, 8, 9, 10\}\\
\indent \indent $\overline{A} \cup \overline{B}$ = \{2, 3, 4, 6, 7, 8, 10\}\\

(b) A $\cup$ B = \{1, 3, 5, 10\}\\
\indent \indent $\overline{A \cup B}$ = \{2, 4, 6, 7, 8, 9\}\\
\indent \indent $\overline{A} \cap \overline{B}$ = \{2, 4, 6, 7, 8, 9\}\\


\hrulefill \\
\pagebreak


\noindent 3)\\
\indent \indent Total probability sums to 1.0, $\therefore$\\
\indent \indent 1k + 2k + 3k + 4k + 5k + 6k = 21k = 1.0\\
\indent \indent k = 1.0 / 21 = 0.0476 $\Rightarrow$\\\\
\indent \indent [1] = 1*0.0476 = 0.0476\\
\indent \indent [2] = 2*0.0476 = 0.0952\\
\indent \indent [3] = 3*0.0476 = 0.1429\\
\indent \indent [4] = 4*0.0476 = 0.1905\\
\indent \indent [5] = 5*0.0476 = 0.2381\\
\indent \indent [6] = 6*0.0476 = 0.2857\\


\hrulefill \\


\noindent 4)\\



\pagebreak
\hrulefill \\

\end{document}
















