\documentclass[12pt]{article}
\usepackage[12pt]{moresize}
\usepackage[margin=1in]{geometry}

\usepackage{amsmath}
\usepackage{amssymb}

\usepackage{graphicx}
\usepackage{subcaption}

\usepackage{multirow} %Combining rows in tables

\usepackage{algorithm}
\usepackage{algpseudocode}
\usepackage{alltt}

\usepackage{multicol}

\usepackage{amssymb} %\checkmark symbol

%\usepackage{hyperref}
%\usepackage[latin1]{inputenc}
%\usepackage{listings}
%\usepackage{scrextend}
%\usepackage{changepage} %Adjustwidth



\title{Stat 330\\Homework 3}
\author{Sean Gordon}
%\date{09/09/2019}

\begin{document}
\maketitle


\noindent\hrulefill \\


\noindent 1)\\
\indent (a) $P(B|A) + P(B|\bar{A})$ = 1
\indent \indent \indent {\Large $\frac{P(B)P(A)}{P(B)} + \frac{P(B)P(\bar{A})}{P(B)}$} = 1\\\\
\indent \indent {\Large $\frac{P(B)P(A) + P(B)P(\bar{A})}{P(B)}$} = 1
\indent \indent \indent  $P(A) + P(\bar{A}) = 1$ \checkmark\\

\indent \indent  \hrulefill \indent \indent \indent \\

\indent (b) $P(\bar{A}|\bar{B}) = 1 - P(A \cup B)
\indent \indent \indent = 1 - P(A) - P(B) + P(A \cap B)\\\\
\indent = 1 - P(A) - P(B) + P(A)P(B)
\indent \indent \indent = (1 - P(A))(1 - P(B))\\\\
\indent  P(\bar{A}|\bar{B}) = P(\bar{A})P(\bar{B})$ \checkmark\\


\noindent \hrulefill \\


\noindent 2)\\
\indent (a) P(A) = .4, \ \ P(B) = .7, \ \ P(A $\cap$ B) = \textbf{.28}\\
\indent (b) P(A$|$B) = P(A $\cap$ B) / P(B) = .28 / .7 = \textbf{.4}\\
\indent (c) P(B$|$A) = P(A $\cap$ B) / P(A) = .28 / .4 = \textbf{.7}\\
\indent (d) A test for independance is if P(A$|$B) = P(A) and P(B$|$A) = P(B)\\
\indent \indent As the above tests hold, these two events are \textbf{independent}.\\


\noindent \hrulefill \\
\pagebreak


\noindent 3)\\
\indent (a) When drawing from the first urn, the chances are:\\
\indent \indent Red: 2/6, \ White: 4/6.\\\\
\indent \indent If a red is transferred, when drawing from urn 2 the chances are: \\
\indent \indent Red: 4/5, \ \textbf{White: 1/5}.\\
\indent \indent If a white is transferred, when drawing from urn 2 the chances are: \\
\indent \indent Red: 3/5, \ \textbf{White: 2/5}.\\\\
\indent \indent Drawing from urn 1 affects drawing from the second, and must be considered.\\
\indent \indent Thus, the cumulative chance of selecting white from urn 2 is:\\ 
\indent \indent {\Large$ \frac{2}{6}*\frac{1}{5} + \frac{4}{6}*\frac{2}{5}$} = \textbf{1/3}\\\\

\indent (b) The two events are not independent, as drawing from the first urn affects the chances\\
\indent \indent  of drawing from the second.\\
\indent \indent Using the values calculated above, \\
\indent \indent P(W2$|$R1) \ = {\Large$ \frac{2}{6}*\frac{1}{5}$} = 1/15\\\\
\indent \indent P(W2$|$W1) = {\Large$ \frac{4}{6}*\frac{2}{5}$} = 4/15\\\\
\indent \indent If the events were independent, the two resulting values would be equivalent.


\noindent \hrulefill \\

\begin{multicols}{2}
\noindent 4)
\begin{tabular}{c|cc|c}
 & $D(.15)$ & $\overline{D}(.85)$ & $Total$ \\
 \hline &&&\\[-1em]
$Pos$ & .98 & .10 & ?\\
$Neg$ & ? & ? & ?\\
 \hline &&&\\[-1em]
 $Total$ & ? & ? & 
\end{tabular}\\

\columnbreak 

\hfill \break
\noindent P(Pos $\cap$ D) = (.98)(.15) = .147\\
P(Pos $\cap \ \overline{D}$) = (.10)(.85) = .085\\

\end{multicols}



\indent (a) P(Pos) = .147 + .085 = \textbf{.232}\\
\indent (b) P(D$|$Pos) = .147 / .232 = \textbf{.634}\\
\indent (c) P($\overline{D}|$Pos) = .085 / .232 = \textbf{.366}\\


\noindent \hrulefill \\


\noindent 5) P(Sys) = P(1st $\cap$ 2nd) \\
\indent P(1st) = 1 - P($\overline{1st}$) = 1 - P($\overline{1} \cap \overline{2} \cap \overline{3}$) = 1 - (.3)(.3)(.3) = 0.973\\
\indent P(2nd) = 1 - P($\overline{2nd}$) = 1 - P($\overline{4} \cap \overline{5}$) = 1 - (.3)(.3) = 0.91\\
\indent P(1st $\cap$ 2nd) = P(1st) * P(2nd) = 0.973 * 0.91 = \textbf{0.885}\\


\noindent \hrulefill 
\pagebreak


\noindent 6)\\
\indent (a) P(Sys) = P(A) * P(B) = .95 * .9 = \textbf{.855}\\

\indent (b) P(Sys) = 1 - P($\overline{Top} \cap \overline{Bottom}$) $\Rightarrow$\\
\indent \indent P(Top) = P(A) * P(B) = .95 * .9 = .855\\
\indent \indent P(Bottom) = P(C) * P(D) = .8 * .7 = .56\\
\indent \indent P(Sys) = 1 - (1 - .855)(1 - .56) = \textbf{.9362}\\

\indent (c) P(Sys) = 1 -  P($\overline{A} \cap \overline{B} \cap \overline{C}$) = 1 - (1 - .95)(1 - .9)(1 - .8) = \textbf{.999}



\end{document}
















