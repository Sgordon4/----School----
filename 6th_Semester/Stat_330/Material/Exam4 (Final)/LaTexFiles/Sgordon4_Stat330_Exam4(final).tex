\documentclass[12pt]{article}
\usepackage[12pt]{moresize}
\usepackage[margin=1in]{geometry}

\usepackage{amsmath}
\usepackage{amssymb}

\usepackage{graphicx}
\usepackage{subcaption}

\usepackage{multirow} %Combining rows in tables
\usepackage{diagbox}  %Table box split in twain

\usepackage{algorithm}
\usepackage{algpseudocode}
\usepackage{alltt}

\usepackage{multicol}

\usepackage{amssymb} %\checkmark symbol

%\usepackage{hyperref}
%\usepackage[latin1]{inputenc}
%\usepackage{listings}
%\usepackage{scrextend}
%\usepackage{changepage} %Adjustwidth

 

\title{Stat 330\\Exam 4 (final)}
\author{Sean Gordon}
\date{May 7, 2020}

\begin{document}
\maketitle


\noindent\hrulefill \\


\noindent 1) \ $Q_1 - 1.5(IQR) = 20 - 1.5(14) = -1$\\
\indent  $Q_3 + 1.5(IQR) = 34 + 1.5(14) = 55$\\
\indent  As $\bf{58}$ is the only number outside the range, it is the only outlier.\\


\noindent \hrulefill \\


\noindent 2)\\
\indent(a) Mean = {\Large $\frac{24 + 31 + 32 + 33 + 35 + 37 + 49}{7}$} = 34.43\\
\indent \indent Median = 33\\
\indent \indent $Q_1 = 31, Q_3 = 37,$ \indent $IQR = Q_3 - Q_1 = 6$\\

\indent (b) $Q_1 - 1.5(IQR) = 31 - 1.5(6) = 22$\\
\indent \indent $Q_3 + 1.5(IQR) = 37 + 1.5(6) = 46$\\
\indent \indent $\bf{49}$ is the only point outside the range.\\

\indent (c) Mean = {\Large $\frac{24 + 31 + 32 + 33 + 35 + 37}{6}$} = 32\\
\indent \indent Median = {\Large $\frac{33 + 32}{2}$} = 31.5\\[.4em]
\indent \indent $Q_1 = 31, Q_3 = 35,$ \indent $IQR = Q_3 - Q_1 = 4$\\


\noindent \hrulefill \\
\pagebreak


\noindent 3) \\
\indent (a) $\hat{\lambda}_1 = (1/3)E(X_1) + (2/3)E(X_2) = (3/3)\theta = \theta$\\
\indent \indent $\hat{\lambda}_2 = E(\bar{X}) = \theta$\\
\indent \indent $\hat{\lambda}_3 = 5$\\
\indent \indent Estimator $\hat{\lambda}_3$ is biased, while $\hat{\lambda}_1$ and $\hat{\lambda}_2$ are unbiased\\

\indent (b) Var($\hat{\lambda}_1$) = (1/9)Var($X_1$) + (4/9)Var($X_2$) = (5/9)$\sigma^2$\\
\indent \indent Var($\hat{\lambda}_2$) = Var($\bar{X}$) = $\sigma^2/n$\\
\indent \indent Var($\hat{\lambda}_3$) = 0\\

\indent (c) MSE($\hat{\lambda}_1$) = $0^2 + (5/9)\sigma^2 = (5/9)\sigma^2$\\
\indent \indent MSE($\hat{\lambda}_2$) = $0^2 + \sigma^2/n = \sigma^2$/n\\
\indent \indent MSE($\hat{\lambda}_3$) = $5^2 - \theta$ + 0 = $5^2 - \theta$\\


\noindent \hrulefill \\


\noindent 4)\\
\indent (a) $\hat{\theta}_{MOM}\Rightarrow ${\Large $\sqrt{\frac{\theta \pi}{2}}$} = $\bar{X}$ $\Rightarrow$ $\theta = ${\Large $\frac{2\bar{X}}{\pi}$}\\
\indent \indent $\bar{X}$ = 2.32 $\Rightarrow$ $\theta = ${\Large $\frac{2*2.23}{\pi}$} = 1.48\\

\indent (b)\\
\indent (i) $log(L(\theta)) = log(x_i\theta^{-n}*e^{-\frac{\Sigma x_i^2}{2\theta}}) = log(x_i)+log(\theta^ 	{-n})+log(e^{-\frac{\Sigma x_i^2}{2\theta}}) = $\\[.4em]
\indent \indent $log(x_i)-nlog(\theta)-${\Large$\frac{\Sigma x_i^2}{2\theta}$}$log(e)$\\[.4em]

\indent (ii) -{\Large$\frac{n}{\theta}$} + {\Large$\frac{\Sigma x_i^2}{2\theta^2}$}log(e)= 0 $\Rightarrow$ {\Large$\frac{\Sigma x_i^2}{2\theta^2}$}log(e) = {\Large$\frac{n}{\theta}$} $\Rightarrow$ $\Sigma x_i^2log(e)\theta = 2n\theta^2$ $\Rightarrow$ $\Sigma x_i^2log(e) = 2n\theta$ $\Rightarrow$ \\[.4em]
\indent \indent \ $ \theta = ${\Large$\frac{\Sigma x_i^2log(e)}{2n}$} $\Rightarrow$ $\hat{\theta}_{mle} = ${\Large$\frac{\Sigma x_i^2log(e)}{2n}$}\\

\indent (iii) $\Sigma x_i^2$ = {\Large$\frac{2.15^2 + 2.68^2 + 2.17^2 + 2.28^2}{4}$} = 5.42\\[.4em]
\indent \indent \ \ $\hat{\theta}_{mle}$ = {\Large$\frac{5.42*0.434}{2(4)}$} = 0.294\\


\noindent \hrulefill \\
\pagebreak


\noindent 5a)\\
\indent (i) $H_0 : \mu = 50$\\
\indent \indent $H_A : \mu \ne 50$\\

\indent (ii) s$_1$ = $\sqrt{64}$ = 8, Z = {\Large $\frac{52 - 50}{8/\sqrt{80}}$} = 2.24\\

\indent (iii) Using Z table, 2*P(Z$<$-2.24) = 2*(0.0125) = 0.025\\
\indent \indent This p value is small, so we reject $H_0$ in favor of $H_A$\\


\noindent 5b)
\indent (i) $H_0 : \mu_1 = \mu_2$\\
\indent \indent $H_A : \mu_1 > \mu_2$\\

\indent (ii) Z = {\Large $\frac{49 - 52 - 0}{\sqrt{ (64/80) + (130/10) }}$} = -.22\\

\indent (iii) Using Z table, P(Z$<$-.22) = 0.4129\\
\indent \indent This p value is not small, so we do not reject $H_0$ in favor of $H_A$\\


\noindent \hrulefill \\


\noindent 6)\\
\indent (a) $.42 \pm 2.326${\Large $\frac{\sqrt{.42(1-.42)}}{7276}$} = .42 $\pm$ .0015 = (.4185, .4215)\\

\indent (b) (.49-.35) $\pm$ 1.96{\Large $\sqrt{\frac{.49(1-.49)}{3638} + \frac{.35(1-.35)}{3638}}$} = .14 $\pm$ .0225 = (.1175, .1625)\\

\indent (c) As we are fairly certain the lowest the difference in proportion goes is .1175, it is safe \\
\indent \indent to say the proportions are not equal.




\end{document}
















