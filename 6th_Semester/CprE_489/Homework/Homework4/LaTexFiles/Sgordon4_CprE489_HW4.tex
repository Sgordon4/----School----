\documentclass[12pt]{article}
\usepackage[12pt]{moresize}

\usepackage{amsmath}
\usepackage{amssymb}

\usepackage{graphicx}
\usepackage{subcaption}

\usepackage{algorithm}
\usepackage{algpseudocode}
\usepackage{alltt}

\usepackage{multicol}

\usepackage{booktabs}
\usepackage{multirow}

\usepackage[margin=1in]{geometry}

%\usepackage{hyperref}
%\usepackage[latin1]{inputenc}
%\usepackage{listings}
%\usepackage{scrextend}
%\usepackage{changepage} %Adjustwidth



\title{CprE 489\\Homework 4}
\author{Sean Gordon}
\date{March 31, 2020}

\begin{document}
\maketitle


\hrulefill \\


\noindent 1a) The vulnerable period covers {\bf t'$_0$-.7X through t'$_0$+1.4X}. The transmission must have one clear slot before t'$_0$ and two clear slots afterward in order to completely fit. Thus the vulnerable period spans 2.1X.\\

\noindent 1b) Max propogation is through 2 repeaters and 2 segments $\Rightarrow$ \\[-2em]
\begin{center}
t$_{prop} = $2(1.5$\mu$s) + 2(100m / (2*10$^8$m/s)) = 3$\mu$s + 1$\mu$s = 4$\mu$s.
\end{center}
This must be multiplied by 2 to ensure the transmitting end receives any response, so frame size = 8$\mu$s.


\hrulefill \\


\noindent 2) \\
\indent (a) 205.63.130.1 AND /16 = 205.63.0.0 \indent 205.63.130.1 AND /18 = 205.63.128.0 \\
\indent \indent 205.63.130.1 AND /21 = 205.63.128.0\\\\
\indent \indent None of these results match any of the destinations listed in the table, so the \\
\indent \indent table will use the default and send the packet to 205.36.1.1\\\\

\indent (b) 205.36.140.2 AND /16 = 205.36.0.0 \indent 205.36.140.2 AND /18 = 205.36.128.0 \\
\indent \indent 205.36.140.2 AND /21 = 205.36.136.0\\\\
\indent \indent Results 2 and 3 match their respective destinations in the table, so as \\
\indent \indent destination 3 has a larger subnet, we send the packet to 205.36.136.1\\

\indent (c) 205.36.150.3 AND /16 = 205.36.0.0 \indent 205.36.150.3 AND /18 = 205.36.128.0 \\
\indent \indent 205.36.150.3 AND /21 = 205.36.144.0\\\\
\indent \indent Only result 2 matches its respective destination, so we send the packet to 205.36.128.1


\hrulefill \\
\pagebreak


\noindent 3) We can divide the given 255 hosts into 3 groups: 96, 96, and 64.\\
\indent  Both groups of 96 must be divided into 2 groups: 64 and 32\\
\indent  This leaves us with the table below:\\


\begin{table}[h!]
\centering
\begin{tabular}{@{}lllll@{}}
\toprule
 	   Department  & IP &&&Size\\ \midrule
 	   	      D1   & 200.120.80.192 & /26 && 64\\[.4em]
\multirow{2}{*}{D2} & 200.120.80.0 & /26 && 64\\
			  & 200.120.80.64 & /27 && 32\\[.4em]
\multirow{2}{*}{D3} & 200.120.80.96 & /26 && 64\\
			  & 200.120.80.160 & /27 && 32\\ \bottomrule

\end{tabular}
\end{table}


\end{document}
















