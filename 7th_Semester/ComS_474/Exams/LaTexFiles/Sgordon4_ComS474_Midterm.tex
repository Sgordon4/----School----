\documentclass[12pt]{article}
\usepackage[12pt]{moresize}
\usepackage[margin=1in]{geometry}

\usepackage{amsmath}
\usepackage{amssymb}

\usepackage{graphicx}
\usepackage{subcaption}

\usepackage{multirow} %Combining rows in tables
\usepackage{diagbox}  %Table box split in twain

\usepackage{algorithm}
\usepackage{algpseudocode}
\usepackage{alltt}

\usepackage{multicol}

\usepackage{amssymb} %\checkmark symbol

%\usepackage{hyperref}
%\usepackage[latin1]{inputenc}
%\usepackage{listings}
%\usepackage{scrextend}
%\usepackage{changepage} %Adjustwidth

 

\title{ComS 474\\Midterm 1}
\author{Sean Gordon}
\date{Oct 18, 2020}

\begin{document}
\maketitle



\noindent I affirm that the work on this exam is my own and I will not use any people to help me nor will I share any
part of this exam or my work with others without permission of the instructor.\\\\\\\\[-.4em]







\noindent 1) Supervised, unsupervised, and Reinforcement learning.\\



\noindent \hrulefill \\



\noindent 2) $w^Tx = 
\begin{pmatrix}
1\\ 2\\ 3
\end{pmatrix}
* (1, 1, 1) = 6 > 1$, so $\hat{y} = 1$\\[-.4em]



\noindent \hrulefill \\



\noindent 3) $(w^Tx-y)^2 = (6 - (-1))^2 = 49$\\[-.4em]



\noindent \hrulefill \\



\noindent 4) As \^{y} can only be $\pm1$, $\sum(\hat{y}-y)^2$ can detect whether the classifier is right or wrong, \\
but loses information on \textbf{how} right or wrong it is. \\
\indent If a classifier predicts a sample's score to be 57 when it really is -1, $(\hat{y}-y)^2$ makes it\\
appear just as wrong as a sample predicted as 2 when it really is -1.\\[-.4em]



\noindent \hrulefill \\\pagebreak



\noindent 5) There are 2 samples with b $>$ 5, and 4 samples with b $<=$ 5. Then...\\[.4em]
\indent $Pr(class = +1 | b>5)$ = 1/2 = 0.5\\
\indent $Pr(class = -1 | b>5)$ = 1/2 = 0.5\\
\indent $Pr(class = +1 | b<=5)$ = 2/4 = 0.5\\
\indent $Pr(class = -1 | b<=5)$ = 2/4 = 0.5\\\\
\indent Using $G(condition)$ = $1 - (Pr(class = +1 | condition))^2 - (Pr(class = -1 | condition))^2$ \\
\indent \indent and the above values...\\[.4em]
\indent $G(b>5)$ = $1 - (0.5)^2 - (0.5)^2$ = 0.5\\
\indent $G(b\le5)$ = $1 - (0.5)^2 - (0.5)^2$ = 0.5\\[-.4em]



\noindent \hrulefill \\



\noindent 6) There are 6 total samples, 2 of which are $>$ 5  and 4 of which are $<=$ 5. Thus...\\
\indent $Pr(b > 5)$ = 2/6 = \textbf{0.333}, and $Pr(b <= 5)$ = 4/6 = \textbf{0.667}\\[-.4em]



\noindent \hrulefill \\



\noindent 7) Expectation = $Pr(b > 5)*G(b > 5)$  +  $Pr(b <= 5)*G(b <= 5)$ \ \ $\Rightarrow$ \\[.4em]
\indent \indent (0.333)(0.5) + (0.667)(0.5) = 0.5\\[-.4em]



\noindent \hrulefill \\



\noindent 8) Any sample with $\lambda > 0$ is considered to be a support vector. Thus, as 3 $\lambda > 0$, 3 of the samples were chosen to be support vectors (samples 1, 3, and 4).\\[-.4em]



\noindent \hrulefill \\



\noindent 9) $
\begin{pmatrix}
w1\\ w2\\ w3
\end{pmatrix}
= \lambda_1 * y_1 * 
\begin{pmatrix}
a_1\\ b_1\\ c_1
\end{pmatrix}
+ \lambda_2 * y_2 * 
\begin{pmatrix}
a_2\\ b_2\\ c_2
\end{pmatrix}
+ \lambda_3 * y_3 * 
\begin{pmatrix}
a_3\\ b_3\\ c_3
\end{pmatrix}
+ \lambda_4 * y_4 * 
\begin{pmatrix}
a_4\\ b_4\\ c_4
\end{pmatrix}$$\Rightarrow$\\[.4em]

$6.13 * (1) * 
\begin{pmatrix}
0.5\\ 0.25\\ 0.125
\end{pmatrix}
+ 0 * (1) * 
\begin{pmatrix}
0.4\\ 0.15\\ 0.225
\end{pmatrix}
+ 4.08 * (-1) * 
\begin{pmatrix}
.3\\ .75\\ .325
\end{pmatrix}
+ 2.05 * (-1) * 
\begin{pmatrix}
0.2\\ 0.65\\ .425
\end{pmatrix}$$\Rightarrow$\\[.4em]

\begin{center}
w = $\begin{pmatrix}
1.431\\ -2.86\\ -1.431
\end{pmatrix}$\\[-.4em]
\end{center}



\noindent \hrulefill \\\pagebreak



\noindent 10) 

(1) 
$\begin{pmatrix}
1.431\\ -2.86\\ -1.431
\end{pmatrix}$ 
* 
$\begin{pmatrix}
0.5\\ 0.25\\ 0.125
\end{pmatrix}$
+ 1.18 = -0.1784 + 1.18 = 1.0016\\[.4em]

(2) 
$\begin{pmatrix}
1.431\\ -2.86\\ -1.431
\end{pmatrix}$ 
* 
$\begin{pmatrix}
0.4\\ 0.15\\ 0.225
\end{pmatrix}$
+ 1.18 = -0.1786 + 1.18 = 1.0014\\[.4em]

(3) 
$\begin{pmatrix}
1.431\\ -2.86\\ -1.431
\end{pmatrix}$ 
* 
$\begin{pmatrix}
0.3\\ 0.75\\ 0.325
\end{pmatrix}$
+ 1.18 = -2.1808 + 1.18 = -1.0008\\[.4em]

(4) 
$\begin{pmatrix}
1.431\\ -2.86\\ -1.431
\end{pmatrix}$ 
* 
$\begin{pmatrix}
0.2\\ 0.65\\ 0.425
\end{pmatrix}$
+ 1.18 = -2.181 + 1.18 = -1.001\\[-.4em]



\noindent \hrulefill \\



\noindent 11) A point is inside the margin when $|wx+w_b| < d$, where d = 1:\\
\indent (1) $|1.0016| > 1$, so this point falls outside of the margin.\\
\indent (2) $|1.0014| > 1$, so this point falls outside of the margin.\\
\indent (3) $|-1.0008| > 1$, so this point falls outside of the margin.\\
\indent (4) $|-1.001| > 1$, so this point falls outside of the margin.\\



\noindent \hrulefill \\



\noindent 12) The value of $w^Tx_i+w_b$ for a support vector should equal $\pm1$, but none of the above values were exactly $\pm1$.



\end{document}

















