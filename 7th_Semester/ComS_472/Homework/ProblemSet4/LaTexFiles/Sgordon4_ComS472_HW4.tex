\documentclass[12pt]{article}
\usepackage[12pt]{moresize}
\usepackage[margin=1in]{geometry}

\usepackage{amsmath}
\usepackage{amssymb}

\usepackage{graphicx}
\usepackage{subcaption}

\usepackage{multirow} %Combining rows in tables
\usepackage{diagbox}  %Table box split in twain

\usepackage{algorithm}
\usepackage{algpseudocode}
\usepackage{alltt}

\usepackage{multicol}

\usepackage{amssymb} %\checkmark symbol

%\usepackage{hyperref}
%\usepackage[latin1]{inputenc}
%\usepackage{listings}
%\usepackage{scrextend}
%\usepackage{changepage} %Adjustwidth

  

\title{ComS 472\\Homework 4}
\author{Sean Gordon}
\date{Oct 26, 2020}

\begin{document}
\maketitle


\centerline{- 7.22 - }
\ \\
\noindent 1) If the pair of clauses has no complimentary literals, there are no resolvents. \checkmark\\[.4em]
\indent If the pair has one or more sets of complimentary literals, the resulting resolvents \\
\indent acquired from applying the same set of literals in any order will eventually reduce\\
\indent down to a single resolvent. \checkmark\\

\noindent 2) See s02\\[.4em]

\noindent 3) For a clause to resolve with a copy of itself, it must contain only complimentary \\[.4em]
\indent literals. This would make the initial clause equivalent to True\\[.4em]



\noindent \hrulefill \\



\centerline{- 7.23 - }
\ \\
\noindent 1) \\[.4em]
\noindent 2) \\[.4em]
\noindent 3) \\[.4em]



\noindent \hrulefill \\



\centerline{- 7.26 - }
\ \\
\noindent S1) \\[.4em]
\noindent S2) \\[.4em]
\noindent S3) \\[.4em]
\noindent S4) \\[.4em]
\noindent S5) \\[.4em]
\noindent S6) \\[.4em]



\noindent \hrulefill \\



\centerline{- 8.11 - }
\ \\
\noindent 1) \\[.4em]
\noindent 2) \\[.4em]
\noindent 3) \\[.4em]
\noindent 4) \\[.4em]
\noindent 5) \\[.4em]
\noindent 6) \\[.4em]
\noindent 7) \\[.4em]



\noindent \hrulefill \\



\centerline{- 8.23 - }
\ \\
\noindent 1) \\[.4em]
\noindent 2) \\[.4em]
\noindent 3) \\[.4em]
\noindent 4) \\[.4em]
\noindent 5) \\[.4em]



\noindent \hrulefill \\



\centerline{- 8.29 - }
\ \\
\noindent 1) \\[.4em]
\noindent 2) \\[.4em]
\noindent 3) \\[.4em]



\noindent \hrulefill \\



\centerline{- 9.4 - }
\ \\
\noindent 1) \\[.4em]
\noindent 2) \\[.4em]
\noindent 3) \\[.4em]
\noindent 4) \\[.4em]



\noindent \hrulefill \\



\centerline{- 9.7 - }
\ \\
\noindent 1) \\[.4em]
\noindent 2) \\[.4em]
\noindent 3) \\[.4em]
\noindent 4) \\[.4em]
\noindent 5) \\[.4em]
\noindent 6) \\[.4em]



\noindent \hrulefill \\



\centerline{- 9.9 - }
\ \\
\noindent 1) \\[.4em]
\noindent 2) \\[.4em]
\noindent 3) \\[.4em]
\noindent 4) \\[.4em]
\noindent 5) \\[.4em]



\noindent \hrulefill \\



\centerline{- 9.16 - }
\ \\
\noindent 1) \\[.4em]
\noindent 2) \\[.4em]
\noindent 3) \\[.4em]



\noindent \hrulefill \\



\centerline{- 9.18 - }
\ \\
\noindent 1) \\[.4em]
\noindent 2) \\[.4em]



\noindent \hrulefill \\



$\neg$$\vee$$\wedge$



\noindent \hrulefill \\


\end{document}

















